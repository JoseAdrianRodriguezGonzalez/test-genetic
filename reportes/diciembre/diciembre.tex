\documentclass{article}
\usepackage[spanish]{babel}
\usepackage{graphicx}
\usepackage[utf8]{inputenc}
\usepackage[spanish]{babel}
\usepackage{natbib}
\usepackage{float}
\usepackage{amsmath}
\usepackage{amssymb}
\usepackage{array}
\usepackage{booktabs}
\title{Clasificación inteligente sinergia entre lógica difusa y algoritmos genéticos }
\author{José Adrián Rodríguez González}
\date{Noviembre 2024}
\begin{document}
\maketitle
\section{Introducción}
Como se menciona en el mes anterior, se implementan las mejoras en el otro problema del cáncer de mama.
\section{Problemática de la ponderación de las clases}
Siempre se ponderaraban las predicciones a una clase específica.

\begin{table}[h!]
    \centering
    \begin{tabular}{@{}cccc@{}}
    \toprule
    \textbf{Regla}       & \textbf{Fitness} & \textbf{CF}    & \textbf{Clase} \\ \midrule
{[0, 0, 0, 0]}    & 20      & 0.2571  & 1     \\
{[3, 1, 0, 0]}    & 2       & 0.4311  & 1     \\
{[0, 1, 4, 1]}    & 0       & 0.4197  & 2     \\
{[4, 5, 1, 1]}    & 0       & 1.0000  & 2     \\
{[1, 1, 0, 2]}    & 0       & 0.9803  & 1     \\
\bottomrule
\end{tabular}
    \caption{Reglas únicas de la Generación 10 con sus métricas correspondientes.}
\label{tab:reglas-unicas}
\end{table}
    La (Tabla \ref{tab:reglas-unicas}), muestra las mejores reglas que salieron del algoritmo genético, como se observa, ya no se pondera a una sola clase, ya que el fitness por lo menos ha estado cambiando y no se estancan en dos únicos valores. Sin embargo, el fitness sigue siendo pequeño.
Estos dos modelos, (Para iris y el del cáncer), pueden ser probados, los conjuntos de reglas, en modelos sintéticos para verificar el accuracy.
El accuracy obtenido fue de 62.7\% un poco cercano a lo que obtuvo \Citep{article1} con el mismo conjunto de datos, sin embargo, este accuracy es más cercano a lo que obtenía con Pittsburgh que con Michigan.Por lo que sería necesario modificar ese primer modelo, además de posiblemente, aplicar una técnica de validación cruzada para ese modelo 
\section{Conclusión}
Como trabajo a futuro, se puede implementar un algoritmo genético híbrido, también se podrían utilizar otros operadores de selección y mutación. Inclusive, para el problema del conjunto de datos de Iris, se puede utilizar el método de Pittsburgh. Otra mejora significativa para el modelo sería la selección del sistema experto a través de lo propuesto por \citep{inbook}. Sin duda, aun queda trabajo de investigación por realizar a este tipo de modelos, debido a lo poco considerados y estudiados que son, además de que el modelo puede irse mejorando con distintas técnicas de algoritmos genéticos.
\bibliographystyle{apalike}
\bibliography{reference}
\end{document}